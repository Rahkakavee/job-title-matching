%%%%%%%%%%%%%%%%%%%%%%%% Title Page %%%%%%%%%%%%%%%%%%%%%%%%
\documentclass[12pt, a4paper, titlepage]{article}

\usepackage{placeins}
\usepackage{float}
\usepackage{hyperref}
\usepackage{graphicx}
\usepackage{appendix}
\usepackage{setspace}
\usepackage{dcolumn}
\usepackage[printonlyused, withpage]{acronym}
\usepackage{amsmath}
\usepackage{natbib}
\makeatletter
\newcommand{\MSonehalfspacing}{%
  \setstretch{1.44}%  default
  \ifcase \@ptsize \relax % 10pt
    \setstretch {1.448}%
  \or % 11pt
    \setstretch {1.399}%
  \or % 12pt
    \setstretch {1.433}%
  \fi
}
\newcommand{\MSdoublespacing}{%
  \setstretch {1.92}%  default
  \ifcase \@ptsize \relax % 10pt
    \setstretch {1.936}%
  \or % 11pt
    \setstretch {1.866}%
  \or % 12pt
    \setstretch {1.902}%
  \fi
}
\makeatother
\MSonehalfspacing


%%%%%%%%%%%%%%DOCUMENT%%%%%%%%%%%%%%
\begin{document}
%%%%%%%%%%%%%%TITLEPAGE%%%%%%%%%%%%%%
\begin{titlepage}
    \begin{center}
    {\LARGE \textbf{Job Title Classification Strategies for the German Labor Market}}
    \\[1cm]
    {\Large \textbf{Masterthesis}}
    \\[1cm]
    {\Large submitted by}
    \\[0.5cm]
    {\LARGE \textbf{Rahkakavee Baskaran}}
    \\[0.5cm]
    {\Large at the}
    \\[0.5cm]
    \includegraphics[width=0.4\textwidth]{logo.jpg}
    \\[1cm]
    {\Large \textbf{Department of Politics and Public Administration}}
    \\[1cm]
    {\Large \textbf{Center for Data and Methods}}
    \\[2cm]
    \begin{minipage}[c]{0.8\textwidth}
    \begin{description}
     \item {\Large \textbf{1.Gutachter:} Prof. Dr. Susumu Shikano}
     \item {\Large \textbf{2.Gutachter:} JunProf Juhi Kulshresthra}
    \end{description}
    \end{minipage}
    \vfill
    {\LARGE \textbf{Konstanz, \today}}
    \end{center}
    \end{titlepage}

%%%%%%%%%%%%%%TableOfContents%%%%%%%%%%%%%%
\tableofcontents
\newpage


%%%%%%%%%%%%%%Abbreviations%%%%%%%%%%%%%%
\section*{Abbreviations}
\begin{acronym}
  \acro{SVM}[SVM]{Support Vector Machine}
\end{acronym}
\newpage

%%%%%%%%%%%%%%SECTIONS%%%%%%%%%%%%%%
\section{Introduction}
\section{Related Work}
\section{Theory}
\section{Data}
\subsection{Baseline Algorithms}
The developed algorithm should be compared against the current state-of-the art methods in order to check the improvements.


CNN-ALGORITHMUS 

However, traditional methods like \ac{SVM} also performed well for text classification. Especially for multiclass tasks, as mentioned in the literature review, often different versions of the algorithm are used and showed good performance \citep{Aiolli2005,Angulo2003,Benabdeslem2006,Guo2015,Mayoraz1999,Tang2019,Tomar2015}. In general \ac{SVM} has several advantages for text classifcation. First, text classifcation usually has a high dimensional input space. \ac{SVM} can handle these large features since they are able to learn independently of the dimensionality of the feature space. In addition \ac{SVM}s are known to perform well for dense and sparse vectors, which is usually the case for text classification \citep{Joachims1998}. Empirical results, for example \citet{Joachims1998} or \cite{Liu2010} confirm the theoretical expectations. It is, therefore, a reasonable option to use a basic version of the \ac{SVM} algorithm as a baseline.

The general idea of a \ac{SVM} is to map ``the input vectors x into a high-dimensional feature space Z through some nonlinear mapping chosen a priori [...], where an optimal separating hyperplane is constructed'' \citep[138]{Vapnik2000}. In \ac{SVM} this optimal hyperplane maximizes the margin, which is simply put the distance from the hyperplane to the closest points, so called Support Vectors, across both classes \citep{Han2012}. Formally, given a training data set with n training vectors $x_i \in R^n, i = 1,....,n$ and the target classes $y_1,...y_i$ with $y_i \in \{-1, 1\}$, the following quadratic programming problem (primal) has to be solved in order to find the optimal hyperplane:
\[\min_{w,b} \frac{1}{2}w^{T}w \] 
\[\text{subject to } y_i(w^T\phi(x_i)+b) \geq 1\]

where $\phi(x_i)$ transforms $x_i$ into a higher dimensional space, $w$ corresponds to the weight and $b$ is the bias \citep{Chang2001,Jordan2006}
The given optimzation function assumes that the data can be separated without errors. This is not always possible, which is why \cite{Cortes1995} introduce a soft margin \ac{SVM}, which allows for missclassfication \citep{Vapnik2000}.
By adding a regularization parameter $C$ with $C > 0$ and the corresponding slack-variable $\xi$ the optimization problem changes to \citep{Chang2001, Han2012}: 
\[\min_{w,b} \frac{1}{2}w^{T}w + C \sum_{i=1}^n \xi_i \] 
\[\text{subject to } y_i(w^T\phi(x_i)+b) \geq 1 - \xi_i, \] 
\[\xi_i \geq, i = 1,...,n\]

Introducing Lagrange multipliers $\alpha_i$ and converting the above optimization problem into a dual problem the optimal $w$ meets \citep{Chang2001, Jordan2006}:
\[w = \sum_{I=1}^n y_i\alpha_i\phi(x_i)\]

with the decision function \citep{Chang2001}:
\[\text{sgn } (w^T\phi(x)+b) = sgn(\sum_{i=1}^n y_i \alpha K(x_i, x) +b)\]

$K(x_i, x)$ corresponds to a Kernel function, which allows to calculate the dot product in the original input space without knowing the exact mapping into the higher space \citep{Han2012, Jordan2006}. 

In order to apply \ac{SVM} to multiclass problems several approaches have been proposed. One stratetgy is to divide the multi-classifcation problem into several binary problems. A common approach here is the one-against-all method. In this method as many \ac{SVM} classifiers are constructed as there are classes k. The k-th classifier assumes that the examples with the k label are positive labels, while all the other examples treated as negative. Another popular approach is the one-against-one method. In this approach $k(k-1)/2$ classifiers are constructed allowing to train in each classifier the data of two classes \citep{Hsu2002}. 

\subsection{Approach}
\section{Results}
\section{Discussion and Limitations}


\clearpage

\bibliographystyle{apalike}
\bibliography{export}
    
\end{document}
